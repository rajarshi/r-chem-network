\documentclass[12pt,letterpaper]{article}
\usepackage{url}
\usepackage{ctable}
\usepackage[onehalfspacing]{setspace}
\usepackage{graphicx}
\usepackage{fullpage, times}

\DeclareGraphicsExtensions{.pdf,.png}
\onehalfspacing

\bibliographystyle{jcics}
\usepackage{anysize}


\begin{document}
\title{Exploring the Role of Small Molecules in Biological Systems Using Network Approaches}
\author{Saurav Das$^\dagger$ and Rajarshi Guha$^\ddagger$\\
$^\dagger$Your Address \\
$^\ddagger$National Center for Advancing Translational Science \\ 9800 Medical Center Drive  Rockville, MD 20850}
\date{}

\maketitle
\begin{abstract}
A nice abstract
\end{abstract}

\section{The Role of Networks in Drug Discovery}
\label{sec:role-networks-drug}

\section{Handling Small Molecules in R}
\label{sec:handl-small-molec}

Small molecule representations are, in general, text based. However,
due to the plethora of chemical formats, they are not natively handled
within R.  Traditionally, one must compute numerical features for
small molecules and perform cheminformatics related manipulation
outside R and import the results of such operations into the R
workspace. An alternative approach is to integrate cheminformatics
toolkits such as the CDK \cite{Steinbeck:2003bh}, RDKit or Indigo into
the R environment. Currently there are two packages that support this
- ChemmineR \cite{Cao:2008fj} and rcdk \cite{Guha:2007aa}.

In this section we briefly review the functionality provided by the
rcdk package, primarily focused on manipulating molecular structures.

\section{Linking Small Molecules to Targets, Pathways and Diseases}
\label{sec:link-small-molec}

\subsection{Drug-target networks}
\label{sec:drug-target-networks}


\subsection{Disease networks}
\label{sec:disease-networks}

\subsection{Dynamic networks}
\label{sec:dynamic-networks}


\section{R as a Platform for Computational Drug Discovery}
\label{sec:r-as-platform}

\section{Summary}
\label{sec:summary}

\bibliography{paper}

\end{document}
